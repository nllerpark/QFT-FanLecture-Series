\documentclass[12pt]{article}
\usepackage{amsmath,amssymb,physics,bm}
\usepackage{graphicx}
\usepackage[dvipsnames]{xcolor}
\usepackage{tcolorbox}
\usepackage[colorlinks=true,
            linkcolor=MidnightBlue,
            urlcolor=RoyalBlue,
            citecolor=OliveGreen]{hyperref}

\title{\textbf{If Dirac Teaches Megumin Quantum Mechanics: A Courtroom Play}}
\author{Junhu Park}
\date{April 14, 2015}

\begin{document}
\maketitle

\section*{Act 1. In the Warp Courtroom}

\textbf{Scene:} The courtroom of Slaanesh in the Warp\cite{SlaaneshWikipedia}.  
Megumin\cite{MeguminWikipedia} is charged with vaporizing three Chaos Marines with her Explosion spell.  
She is aware of her actions but has not confessed in court.

Paul Dirac\cite{DiracWikipedia} appears as her defense attorney.  
He argues that, by the principle of quantum observation\cite{neumann1955mathematical}, guilt and innocence should be treated as a superposition until a measurement is made.

\textbf{Co-defendant:}  
Aqua\cite{AquaWikipedia} is present in the defendant's box due to sheer misfortune, still clueless about the situation.

\begin{tcolorbox}[title=Dirac:]
If the universe follows the laws of quantum mechanics\cite{dirac1981principles,griffiths2018introduction}, Megumin, your state is not determined until it is observed. That is, your guilt $\ket{G}$ and innocence $\ket{I}$ are in superposition:
\[
\ket{\Psi} = \alpha \ket{G} + \beta \ket{I}, \quad \text{with } |\alpha|^2 + |\beta|^2 = 1.
\]
\end{tcolorbox}

\section*{Quantum Interlude I: Quantization of Verdicts}

\textbf{Dirac (narration):}

The verdict operator of the court is quantized\cite{nielsen2010quantum} as follows:
\[
\hat{H}_{\text{Justice}} \ket{G} = +1 \ket{G}, \quad \hat{H}_{\text{Justice}} \ket{I} = 0 \ket{I}
\]
This means verdicts are not emotional gradients, but discrete eigenvalues. One cannot define innocence without defining guilt. That is the quantum identity\cite{dirac1981principles}.

\section*{Act 2. The Interference of Guilt}

\begin{tcolorbox}[title=Dirac:]
If the phase is aligned, constructive interference appears:
\[
\ket{\Psi} = \frac{1}{\sqrt{2}}(\ket{G} + \ket{I}),
\]
and if the phase is opposite,
\[
\ket{\Psi} = \frac{1}{\sqrt{2}}(\ket{G} - \ket{I}),
\]
then destructive interference reduces the observable probability of guilt\cite{griffiths2018introduction}.
\end{tcolorbox}

\section*{Interlude II: Uncertainty of Verdict and Sentence}

\textbf{Dirac (aside):}

If the guilt operator $\hat{G}$ and the sentence operator $\hat{P}$ do not commute\cite{griffiths2018introduction},
\[
[\hat{G}, \hat{P}] \neq 0 \Rightarrow \Delta G \, \Delta P \geq \frac{\hbar}{2}
\]
then a precise verdict implies uncertainty in sentencing, and vice versa.

\section*{Act 3. Collapse or Not Collapse}

\[
\hat{P}_{G} = \ket{G}\bra{G}, \quad \hat{P}_{I} = \ket{I}\bra{I}
\]

\begin{tcolorbox}[title=Dirac:]
If you consent to measurement, the wavefunction will collapse\cite{neumann1955mathematical}, and reality will choose a single branch.
\end{tcolorbox}

\begin{tcolorbox}[title=Megumin:]
...I accept the observation. Just as I always chose Explosion.
\end{tcolorbox}

\section*{Act 4. Entanglement and Wigner's Friend}

\begin{tcolorbox}[title=Judge Slaanesh:]
Measurement result: $\ket{I}$. Not guilty.
\end{tcolorbox}

\begin{tcolorbox}[title=Dirac:]
Her guilt component vanished into orthogonality, and Aqua became entangled with the $\ket{G}$ state\cite{nielsen2010quantum}.
\end{tcolorbox}

\textbf{Dirac (internal monologue):}

Even if Megumin knew she was guilty, unless she declares it, no external observation occurs. Hence, the wavefunction remains uncollapsed. This is the puzzle left by Wigner's friend\cite{zurek2002decoherence}.

\section*{Interlude III: State Reconstruction}

\textbf{Dirac (narration):}

If the court reconstructs the density matrix $\rho$ from testimonies\cite{nielsen2010quantum},
\[
\rho = \sum_{i,j} p_{ij} \ket{i}\bra{j},
\]
then this is quantum state tomography. Truth is reconstructed from observation\cite{zurek2002decoherence}.

\section*{Act 5. ...}

\textit{(Empty stage. The curtain falls.)}

\textbf{Note:} An unobserved world is neither real nor unreal. The verdict in quantum court is always open, and the final question of physics remains: \emph{Who observed it?}\cite{carroll2004introduction}

\newpage

\bibliographystyle{plain}
\bibliography{refs}


\end{document}
